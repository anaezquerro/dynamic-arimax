\documentclass[10pt, aspectratio=169]{beamer}
\usetheme{Boadilla}
\usecolortheme{seahorse}
\usepackage{amssymb,amsmath}
\usepackage[utf8]{inputenc}

\usepackage{xcolor}
\usepackage{hyperref}
\usepackage{multicol}



% beamer style 

\definecolor{xoveticblue}{HTML}{3cafc2} % UBC Blue (primary)
\definecolor{xoveticorange}{HTML}{e58923}
\definecolor{xoveticdarkorange}{HTML}{d97b27}
\hypersetup{
    colorlinks = true,
    linkcolor = xoveticblue,
    urlcolor = xoveticblue
}


\setbeamercolor{palette primary}{bg=xoveticorange,fg=white}
\setbeamercolor{palette secondary}{bg=xoveticblue,fg=white}
\setbeamercolor{palette tertiary}{bg=xoveticdarkorange, fg=white}
\setbeamercolor{palette quaternary}{bg=xoveticorange, fg=white}
\setbeamercolor{structure}{fg=xoveticorange} % itemize, enumerate, etc
\setbeamercolor{section in toc}{fg=xoveticorange} % TOC sections
\setbeamercolor{subsection in head/foot}{bg=xoveticorange,fg=white}



\usepackage[capitalise, noabbrev, nameinlink]{cleveref}
\usepackage{float}
\usepackage{booktabs}
\usepackage{makecell}
\setbeamertemplate{caption}[numbered]

% -------- Portada ---------
\title[\textcolor{white}{Automatic covariates selection in DRM}]{Automatic covariates selection in dynamic regression models with application to COVID-19 evolution}

\author[Ana Ezquerro, Germán Aneiros, Manuel Oviedo]{
    Ana Ezquerro \inst{1} 
    \and 
    Germán Aneiros \inst{2}
    \and 
    Manuel Oviedo \inst{3}
}
\institute[]{
   \inst{1} University of A Coruña, \texttt{ana.ezquerro@udc.es}
\and
    \inst{2} CITIC, Grupo MODES, Departamento de Matemáticas, University of A Coruña, \texttt{german.aneiros@udc.es}
\and
    \inst{3} CITIC, Grupo MODES, Departamento de Matemáticas, University of A Coruña, \texttt{manuel.oviedo@udc.es}
}

\date{XoveTIC 2022 Conference}
\titlegraphic{\vspace{-1.2em}\includegraphics[scale=0.4]{gallery/udc.png}
\includegraphics[scale=0.3]{gallery/modes.png}
\includegraphics[scale=0.5]{gallery/citic.png}
}


\begin{document}

\frame{\titlepage}

\begin{frame}
    \frametitle{Introduction}
        The \textbf{linear dynamic regression model} defines the linear dependence between a stochastic process $Y_t$ and a set of processes $\mathcal{X} =  \{ X_t^{(1)}, ..., X_t^{(m)} \}$:

        \[  Y_t = \beta_0 + \beta_1 X_{t-r_1}^{(1)} + \cdots \beta_m X_{t-r_m}^{(m)} + \eta_t
        \]

        constrained to $r_i\geq 0$ for $i=1,...,m$ and $\eta_t \sim \text{ARMA}(p,q)$\footnote{here we denote the AutoRegressive Moving Average model as ARMA.}.



        \begin{itemize}
            \item \cite{cryer2008time} proposed the \textit{prewhitening} as a technique for removing spurious correlation between processes in order to detect linear correlation.
            \item We propose a forward-selection method that iteratively adds regressor variables (from a set of candidates) $Y_t$ is \textit{significantly} dependent with.
        \end{itemize}

\end{frame}

\begin{frame}{Example of spurious correlation and prewhitening}
    \begin{figure}
        \includegraphics[scale=0.35]{gallery/example prewhitening.pdf}
    \end{figure}
    
\end{frame}

\begin{frame}{Methodology}
    

    Let $Y_t$ be the dependent variable and $\mathcal{X}$ the set of covariates candidates. Thus, selection proceeds as follows:
    
    \begin{enumerate}
        \item Initialization. Consider $X_t^\text{best}$ as the covariate (lagged $r$ moments) that minimizes the IC of the constructed model with $Y_t$:
        \[ X_t^\text{best} = \underset{X_t\in\mathcal{X}}{\arg\min} \Bigg\{ \text{IC}\Bigg( Y_t = \beta_0 + \beta_1 X_{t-r}^\text{best} + \eta_t \Bigg)\Bigg\} \] 
        \item Iteration. Use the regression errors ($\eta_t$) of the last model created to check if some correlation exists between the rest of the covariates not yet added to the model. Find again the ``best'' variable and add it to the model to obtain a new IC value. If this value improves the last one achieved, repeat this step. If it does not, stop the iteration.
        \item Finalization. The errors of the last fitted model must satisfy the stationary property. In other case, consider the regular differentiation of all data and start again the procedure.
    \end{enumerate}
\end{frame}


\begin{frame}{Simulation procedure}
    \begin{enumerate}
        \item Seven time series (modelable by an ARIMA) were generated: six act as covariate candidates $\mathcal{X}=\{X_t^{(1)}, ..., X_t^{(6)}\}$ with random lags $r\in[0, 6]$, and the remaining as the errors of the model.
        \item Formally, each simulation follows this scheme:
        \[ Y_t = \beta_0 + \beta_1 X_{t-r_1}^{(1)} + \beta_2 X_{t-r_2}^{(2)} + \beta X_{t-r_3}^{(3)} + \eta_t \] 
        where $\eta_t\sim\text{ARMA}(p,q)$, $\beta_0,...,\beta_3$ are randomly generated and $r_i\in[0,6] $ for $i=1,..,3$.
        \item Selection method was tested with different configurations:
        \begin{itemize}
            \item Changing the IC with three different options: AIC, BIC or AICc.
            \item Changing the method to check stationary: via the Dickey-Fuller test or via adjusting an ARIMA and checking the differentiation order.
        \end{itemize}
    \end{enumerate}
    
\end{frame}

\begin{frame}{Example of one simulation result}
    \begin{figure}
        \caption{Code output when launching the function \texttt{drm.select()} that implements our approach}
        \includegraphics[scale=0.5]{gallery/simulation-example.pdf}
    \end{figure}
    
\end{frame}


\begin{frame}{Results of multiple simulations where $\eta_t\sim\text{ARMA(p,q)}$}

    \begin{table}
        \centering\small 
        \renewcommand{\arraystretch}{1.3}
        \setlength{\tabcolsep}{10pt}
        \label{simulations1}
        \caption{Percentage data results when residuals are stationary}
        \begin{tabular}{|c|ccc|ccc|}
            \cline{2-7}
            \multicolumn{1}{c|}{}   & \textbf{AIC}  & \textbf{BIC}  & \textbf{AICc}            & \textbf{AIC}      & \textbf{BIC}   & \textbf{AICc} \\
            \hline       
            \textbf{adf.test}       & 97.66\%       & 97.66\%        & 97.66\%                 & 3.66\%            & 1.33\%         & 3.66\%        \\
            \textbf{auto.arima}     & 98.33\%       & 98.33\%        & 98.33\%                 & 3.66\%            & 1.33\%         & 3.66\%        \\
            \hline
            \multicolumn{1}{|c|}{}   & \multicolumn{3}{c|}{correctly added (TP)}               & \multicolumn{3}{c|}{incorrectly added (FP)}         \\
            \hline 
            \multicolumn{7}{c}{}                                                                                                                    \\
            \cline{2-7}
            \multicolumn{1}{c|}{}   & \textbf{AIC}  & \textbf{BIC}  & \textbf{AICc}           & \textbf{AIC}      & \textbf{BIC}   & \textbf{AICc}  \\ 
            \hline        
            \textbf{adf.test}       & 96.33\%       & 98.66\%        & 96.33\%                & 2.33\%            & 2.33\%         & 2.33\%         \\
            \textbf{auto.arima}     & 96.33\%       & 98.66\%        & 96.33\%                & 1.66\%            & 1.66\%         & 1.66\%         \\
            \hline 
            \multicolumn{1}{|c|}{}   & \multicolumn{3}{c|}{correctly \textbf{not} added (TN)} & \multicolumn{3}{c|}{incorrectly \textbf{not} added (FN)}  \\
            \hline
        \end{tabular}
    \end{table}

\end{frame}

\begin{frame}{Results of multiple simulations where $\eta_t\sim\text{ARIMA(p,d,q)}$}
    \begin{table}
        \centering\small 
        \renewcommand{\arraystretch}{1.3}
        \setlength{\tabcolsep}{10pt}
        \label{simulations2}
        \caption{Percentage data results when residuals are non-stationary}
        \begin{tabular}{|c|ccc|ccc|}
            \cline{2-7}
            \multicolumn{1}{c|}{}   & \textbf{AIC}  & \textbf{BIC}  & \textbf{AICc}            & \textbf{AIC}      & \textbf{BIC}   & \textbf{AICc} \\
            \hline       
            \textbf{adf.test}       & 93.33\%       & 93.33\%        & 93.33\%                 & 4.33\%            & 0.30\%         & 4.33\%        \\
            \textbf{auto.arima}     & 94.33\%       & 94.66\%        & 95.33\%                 & 5.00\%            & 1.33\%         & 5.00\%        \\
            \hline
            \multicolumn{1}{|c|}{}   & \multicolumn{3}{c|}{correctly added (TP)}               & \multicolumn{3}{c|}{incorrectly added (FP)}         \\
            \hline 
            \multicolumn{7}{c}{}                                                                                                                    \\
            \cline{2-7}
            \multicolumn{1}{c|}{}   & \textbf{AIC}  & \textbf{BIC}  & \textbf{AICc}           & \textbf{AIC}      & \textbf{BIC}   & \textbf{AICc}  \\ 
            \hline        
            \textbf{adf.test}       & 95.00\%       & 98.66\%        & 95.00\%                & 6.66\%            & 6.66\%         & 6.66\%         \\
            \textbf{auto.arima}     & 94.66\%       & 99.66\%        & 95.66\%                & 4.66\%            & 5.33\%         & 4.66\%         \\
            \hline 
            \multicolumn{1}{|c|}{}   & \multicolumn{3}{c|}{correctly \textbf{not} added (TN)} & \multicolumn{3}{c|}{incorrectly \textbf{not} added (FN)}  \\
            \hline
        \end{tabular}
    \end{table}
\end{frame}


\begin{frame}{Application to COVID19 evolution}
    \begin{table}
        \centering
        \setlength{\tabcolsep}{10pt}
        \caption{Information about the dynamic regression model constructed via selection of multiple vaccination variables to model COVID19 evolution} 
        \label{covid19model}
    
        \vspace{0.5em}
        \begin{tabular}{|l|cc|}
            \hline
            \textbf{Covariate}  & \textbf{Lag}  & \textbf{Coefficient est. (s.e)} \\ 
            \hline 
            \texttt{vac4565}    & -3            & -0.0410 (0.0057)                      \\ 
            \texttt{vac6580}    & -2            & -0.0468 (0.0120)                      \\
            \texttt{vac1845}    & -6            & -0.0901 (0.0047)                      \\
            \hline
            \texttt{vac1218}    & \multicolumn{2}{c|}{Not included in the model} \\
            \texttt{vac80}      & \multicolumn{2}{c|}{Not included in the model} \\
            \hline
            residuals           & ARIMA(4, 0, 0) & \makecell[c]{$\phi_1=2.0816 (0.0810)$ \\ $\phi_2=-1.2837 (0.1152)$ \\ $\phi_4=0.1919 (0.0432)$ } \\
            \hline
        \end{tabular}
    \end{table}    
    \begin{itemize}\small
        \item Lagged negative correlation between the vaccination data and COVID-19 evolution.
        \item Vaccination of young population (from 18 up to 45 years old) has a greater impact in the COVID19 evolution.
        \item Vaccination of the youngest and oldest range of ages has no significative impact in the COVID19 confirmed cases.
    \end{itemize}
    
\end{frame}

\begin{frame}{Conclusions and future work}
    \begin{itemize}
        \item R implementation is openly available in: 
        \begin{center}
            \url{https://github.com/anaezquerro/dynamic-arimax}
        \end{center}
        \item This code has been optimized and some steps are parallelized.
        \item Widespread application in financial, economic and biomedical fields.
        \item Other covariates might be considered, such as discrete or functional variables.
    \end{itemize}

    \begin{block}{Thanks to}
        \begin{itemize}
            \item Department Collaboration Scholarship financed by Banco Santander.
            \item Mathematics Department of the University of A Coruña.
        \end{itemize}
    \end{block}
\end{frame}


\begin{frame}
    \bibliography{bibliography}
    \bibliographystyle{apalike}
\end{frame}

\end{document}  