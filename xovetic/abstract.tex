\documentclass[12pt, a4paper, twoside]{article}

% ---------------- Configuración estilo -----------------------
\usepackage[utf8]{inputenc}
\usepackage[spanish,es-nodecimaldot]{babel}
\usepackage{sansmathfonts}
\renewcommand*\familydefault{\sfdefault}

% -------------- Configuración de hiperenlaces -----------------
\usepackage{hyperref}
\hypersetup{
    colorlinks = true,
    linkcolor = blue,
    citecolor = blue,
    urlcolor = blue,
}

% ----------------- Configuración de la página --------------------
\usepackage{geometry}
\geometry{
    top         	= 1.2in,
    left			= 1in,
    right			= 1in,
    bottom			= 1.2in,
    headsep			= 20pt,
    footnotesep		= 20pt
}


% ------------ Configuración de los párrafos y enumeraciones ---------
\setlength{\parskip}{0.7em}
\setlength{\parindent}{0em}
\usepackage[shortlabels]{enumitem}
\setlist[enumerate]{leftmargin=0.7cm, topsep=4pt, parsep=2pt}
\setlist[itemize]{leftmargin=0.7cm, topsep=4pt, parsep=2pt}


% ------------------ Paquetes matemáticos -----------------------
\usepackage{xfrac}
\usepackage{amsmath}
\usepackage{amssymb}
\usepackage{mathtools}
\usepackage{bm}
\setlength{\abovedisplayskip}{1pt} 
\setlength{\belowdisplayskip}{1pt}

\usepackage[capitalise, noabbrev, nameinlink]{cleveref}


% ----------------- Paquetes de diseño -------------------------
\usepackage{tabto}
\usepackage{array}
\usepackage{float}
\usepackage{pifont}
\usepackage{soul}
\usepackage{fancyvrb}
\usepackage{longtable, supertabular, ltablex, booktabs}
\usepackage[cache=false]{minted}

\renewcommand{\theFancyVerbLine}{\sffamily
\textcolor[rgb]{0.5,0.5,1.0}{\scriptsize
\oldstylenums{\arabic{FancyVerbLine}}}}

% ------------------ Configuración de las tablas -----------------
\usepackage{makecell}
\usepackage{multicol}
\usepackage{multirow}
\usepackage{tabularx}
\usepackage{caption} 
\usepackage{diagbox}
\usepackage{graphicx}
\captionsetup[table]{skip=10pt}

% ---------------- Definición de colores ----------------------
\usepackage{xcolor}
\definecolor{chaptercolor}{HTML}{3725D9}
\definecolor{sectioncolor}{HTML}{3453F0}
\definecolor{subsectioncolor}{HTML}{2572D9}
\definecolor{referencescolor}{HTML}{D589F5}
\definecolor{pastelgreen}{HTML}{9FF0C0}
\definecolor{pastelorange}{HTML}{FFF0E3}
\definecolor{pastelpurple}{HTML}{E2D4FF}
\definecolor{pastelblue}{HTML}{C6D1FF}
\definecolor{pastelcyan}{HTML}{BAECFF}
\definecolor{backgreen}{HTML}{C7F0DA}
\definecolor{LightGray}{gray}{0.9}
\definecolor{cellborder}{HTML}{CFCFCF}
\definecolor{cellbackground}{HTML}{F7F7F7}
\newcommand{\hlg}[2]{{\sethlcolor{#1} \hl{#2}}}
\newcommand{\hlgreen}[1]{{\sethlcolor{pastelgreen} \hl{#1}}}
\newcommand{\hlblue}[1]{{\sethlcolor{pastelblue} \hl{#1}}}
\newcommand{\hlcyan}[1]{{\sethlcolor{pastelcyan} \hl{#1}}}

% ---------------- Configuración de las cajas -----------------
\usepackage[most, breakable]{tcolorbox}
\tcbuselibrary{skins}
\definecolor{boxborder}{HTML}{8DA6FF}
\definecolor{boxbackground}{HTML}{E2F6F3}
\tcbset{
    enhanced,
    colback=white,
    colframe=boxborder,
    boxrule=1pt, 
    coltitle=black, 
    toptitle=1mm, 
    drop shadow,
    bottomtitle=1mm, 
    fonttitle=\bfseries,
    arc=2mm,
}
\newtcolorbox{mathbox}[1][]
{
    capture=hbox,
    halign=flush center,
    math,
    #1,
}

\newtcolorbox{codebox}[1][]
{
    enhanced, 
    breakable,
    colback=cellbackground,
    colframe=cellborder,
    arc=0mm,
    #1,
}


% -------------- Configuración de los títulos  -------------------
\usepackage{titlesec}
\titleformat*{\section}{\bfseries\large}
\titlespacing*{\section}{0pt}{15pt}{7pt}
\titleformat*{\subsection}{\normalfont}
\titlespacing*{\subsection}{0pt}{10pt}{5pt}


% -------------------- Configuración portada ---------------------
\title{\textbf{Selección automática de variables regresoras en modelos de regresión lineal dinámica con aplicaciones al COVID-19}}
\author{Ana Xiangning Pereira Ezquerro \\ \texttt{ana.ezquerro@udc.es} \and 
        Germán Aneiros Pérez \\ \texttt{german.aneiros@udc.es} \and 
        Manuel Oviedo de la Fuente \\ \texttt{manuel.oviedo@udc.es}}
\date{Departamento de Matemáticas, Curso 2021-2022}

% ---------------- Configuración de encabezados y pies ------------
\usepackage{fancyhdr}
\pagestyle{fancy}
\fancyhf{}
\fancyhead[LE, RO]{\footnotesize \rightmark}
\fancyhead[RE, LO]{\footnotesize Ana Xiangning Pereira Ezquerro}
\fancyfoot{}
\fancyfoot[RE, LO]{\footnotesize \textcolor{chaptercolor}{Congreso XoveTIC}}
\fancyfoot[RO, LE]{\footnotesize \thepage}
\renewcommand{\headrulewidth}{0.5pt}
\renewcommand{\footrulewidth}{0.5pt}

\begin{document}
\maketitle



En el estudio y análisis de series temporales, los bien conocidos modelos de regresión dinámica permiten modelar formalmente la dependencia entre un conjunto de variables regresoras y una variable dependiente teniendo en cuenta la componente temporal inherente a todas las variables participantes. En base al trabajo previo de \cite{cryer2008time} y \cite{hyndman2018forecasting}, se propone un método iterativo de tipo \textit{forward selection} para añadir variables significativas de un conjunto dado al modelo de regresión con sus respectivos retardos óptimos.

Formalmente, los modelos de regresión lineal dinámica expresan la dependencia lineal entre un proceso $Y_t$ y un conjunto de procesos $\mathcal{X}=\{ X_t^{(1)}, X_t^{(2)}, ..., X_t^{(m)}\}$ en instantes no superiores a $t$:

\[ Y_t = \beta_0 + \beta_1 X^{(1)}_{t-r_1} + \beta_2 X^{(2)}_{t-r_2} + \cdots +  \beta_m  X^{(m)}_{t-r_m} + \eta_t \]

donde $r_i \geq 0$, para $i=1,...,m$, y $\eta_t \sim$ ARMA(p,q).

\cite{cryer2008time} propuso un método denominado \textit{preblanqueado} para eliminar la correlación espuria entre dos procesos $X_t$ e $Y_t$ y así detectar de forma limpia la existencia de dependencia lineal entre ellos, así como el retardo $r$ en el que se produce dicha dependencia. Siguiendo su metodología, esta nueva aproximación permite añadir de forma iterativa procesos dependientes al modelo en base a chequear si existe una correlación significativa (siguiendo el método de \cite{cryer2008time}) entre un nuevo proceso y los residuos $\eta_t$ del modelo.

Dado un proceso $Y_t$ y un conjunto de procesos estocásticos $\mathcal{X}=\{ X_t^{(1)}, X_t^{(2)}, ..., X_t^{(m)}\}$ que actúan como variables regresoras en el modelo, y un criterio de información para evaluar el modelo, el procedimiento a seguir es:

\begin{enumerate}
    \item Buscar en todo el conjunto $\mathcal{X}$ el proceso con su retardo óptimo que mejor modelo de regresión lineal dinámica simple produzca, en base al criterio de información escogido. 
    \[ X^{\text{best}}_t = \underset{X\in\mathcal{X}}{\arg\min} \Big\{ \text{criteria}\big( Y_t = \beta_0 + \beta_1 X_{t-r} + \eta_t\big)\Big\} \] 
    \item Si $X^{\text{best}}_t$ existe, construir el modelo $\mathcal{M}: Y_t =  \beta_0 + \beta_1 X^{\text{best}}_{t-r} + \eta_t$, retirar $X^{\text{best}}$ de $\mathcal{X}$, tomar $\tilde{Y}_t=\eta_t$ y proceder de forma iterativa:
    \begin{enumerate}[label*=\arabic*.]
        \item Buscar en $\mathcal{X}$ el proceso  $X^{\text{best}}_t$ tal que:
        \[ X^{\text{best}}_t = \underset{X\in\mathcal{X}}{\arg\min} \Big\{ \text{criteria}\big( \tilde{Y}_t = \beta_0 + \beta_1 X_{t-r} + \eta_t\big)\Big\} \] 
        con la condición de que $\text{criteria}\big(\tilde{Y}_t = \beta_0 + \beta_1 X_{t-r} + \eta_t\big) < \text{criteria}(\mathcal{M})$
        \item En caso de que se pueda obtener dicho modelo, considerar un nuevo $\mathcal{M}: \tilde{Y}_t = \beta_0 + \beta_1 X^{\text{best}}_t + \eta_t$ y $\tilde{Y}_t = \eta_t$ y volver a (2.1). En otro caso, detener el algoritmo.
    \end{enumerate}
    \item El modelo de regresión lineal dinámica se construye a partir de  $Y_t$ y el conjunto de procesos escogidos como $ X^{\text{best}}_t$ en cada iteración del algoritmo.
\end{enumerate}

Esta nueva propuesta ha sido implementada y optimizada en el lenguaje estadístico \href{https://www.r-project.org/}{R} en formato de paquete, y se ha aplicado sobre una serie de simulaciones para validar su rendimiento. Finalmente se presentan los resultados obtenidos usando la base de datos de IRAS de Cataluña para analizar la evolución de la COVID-19. 



\bibliography{bibliography}
\bibliographystyle{apalike}

\end{document}

