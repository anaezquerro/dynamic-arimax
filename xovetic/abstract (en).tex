\documentclass[12pt, a4paper, twoside]{article}

% ---------------- Configuración estilo -----------------------
\usepackage[utf8]{inputenc}
\usepackage[english]{babel}
\usepackage{sansmathfonts}
\renewcommand*\familydefault{\sfdefault}

% -------------- Configuración de hiperenlaces -----------------
\usepackage{hyperref}
\hypersetup{
    colorlinks = true,
    linkcolor = blue,
    urlcolor = blue,
    citecolor = blue
}

% ----------------- Configuración de la página --------------------
\usepackage{geometry}
\geometry{
    top         	= 1.2in,
    left			= 1in,
    right			= 1in,
    bottom			= 1.2in,
    headsep			= 20pt,
    footnotesep		= 20pt
}


% ------------ Configuración de los párrafos y enumeraciones ---------
\setlength{\parskip}{0.7em}
\setlength{\parindent}{0em}
\usepackage[shortlabels]{enumitem}
\setlist[enumerate]{leftmargin=0.7cm, topsep=4pt, parsep=2pt}
\setlist[itemize]{leftmargin=0.7cm, topsep=4pt, parsep=2pt}


% ------------------ Paquetes matemáticos -----------------------
\usepackage{xfrac}
\usepackage{amsmath}
\usepackage{amssymb}
\usepackage{mathtools}
\usepackage{bm}
\setlength{\abovedisplayskip}{1pt} 
\setlength{\belowdisplayskip}{1pt}

\usepackage[capitalise, noabbrev, nameinlink]{cleveref}


% ----------------- Paquetes de diseño -------------------------
\usepackage{tabto}
\usepackage{array}
\usepackage{float}
\usepackage{pifont}
\usepackage{soul}
\usepackage{fancyvrb}
\usepackage{longtable, supertabular, ltablex, booktabs}
\usepackage[cache=false]{minted}

\renewcommand{\theFancyVerbLine}{\sffamily
\textcolor[rgb]{0.5,0.5,1.0}{\scriptsize
\oldstylenums{\arabic{FancyVerbLine}}}}

% ------------------ Configuración de las tablas -----------------
\usepackage{makecell}
\usepackage{multicol}
\usepackage{multirow}
\usepackage{tabularx}
\usepackage{caption} 
\usepackage{diagbox}
\usepackage{graphicx}
\captionsetup[table]{skip=10pt}

% ---------------- Definición de colores ----------------------
\usepackage{xcolor}
\definecolor{chaptercolor}{HTML}{3725D9}
\definecolor{sectioncolor}{HTML}{3453F0}
\definecolor{subsectioncolor}{HTML}{2572D9}
\definecolor{referencescolor}{HTML}{D589F5}
\definecolor{pastelgreen}{HTML}{9FF0C0}
\definecolor{pastelorange}{HTML}{FFF0E3}
\definecolor{pastelpurple}{HTML}{E2D4FF}
\definecolor{pastelblue}{HTML}{C6D1FF}
\definecolor{pastelcyan}{HTML}{BAECFF}
\definecolor{backgreen}{HTML}{C7F0DA}
\definecolor{LightGray}{gray}{0.9}
\definecolor{cellborder}{HTML}{CFCFCF}
\definecolor{cellbackground}{HTML}{F7F7F7}
\newcommand{\hlg}[2]{{\sethlcolor{#1} \hl{#2}}}
\newcommand{\hlgreen}[1]{{\sethlcolor{pastelgreen} \hl{#1}}}
\newcommand{\hlblue}[1]{{\sethlcolor{pastelblue} \hl{#1}}}
\newcommand{\hlcyan}[1]{{\sethlcolor{pastelcyan} \hl{#1}}}

% ---------------- Configuración de las cajas -----------------
\usepackage[most, breakable]{tcolorbox}
\tcbuselibrary{skins}
\definecolor{boxborder}{HTML}{8DA6FF}
\definecolor{boxbackground}{HTML}{E2F6F3}
\tcbset{
    enhanced,
    colback=white,
    colframe=boxborder,
    boxrule=1pt, 
    coltitle=black, 
    toptitle=1mm, 
    drop shadow,
    bottomtitle=1mm, 
    fonttitle=\bfseries,
    arc=2mm,
}
\newtcolorbox{mathbox}[1][]
{
    capture=hbox,
    halign=flush center,
    math,
    #1,
}

\newtcolorbox{codebox}[1][]
{
    enhanced, 
    breakable,
    colback=cellbackground,
    colframe=cellborder,
    arc=0mm,
    #1,
}


% -------------- Configuración de los títulos  -------------------
\usepackage{titlesec}
\titleformat*{\section}{\bfseries\large}
\titlespacing*{\section}{0pt}{15pt}{7pt}
\titleformat*{\subsection}{\normalfont}
\titlespacing*{\subsection}{0pt}{10pt}{5pt}


% -------------------- Configuración portada ---------------------
\title{\textbf{Automatic covariates selection in dynamic regression models with application to COVID-19 evolution}}
\author{Ana Xiangning Pereira Ezquerro \\ \texttt{ana.ezquerro@udc.es} \and 
        Germán Aneiros Pérez \\ \texttt{german.aneiros@udc.es} \and 
        Manuel Oviedo de la Fuente \\ \texttt{manuel.oviedo@udc.es}}
\date{Departamento de Matemáticas, Curso 2021-2022}

% ---------------- Configuración de encabezados y pies ------------
\usepackage{fancyhdr}
\pagestyle{fancy}
\fancyhf{}
\fancyhead[LE, RO]{\footnotesize \rightmark}
\fancyhead[RE, LO]{\footnotesize Ana Xiangning Pereira Ezquerro}
\fancyfoot{}
\fancyfoot[RE, LO]{\footnotesize \textcolor{chaptercolor}{Congreso XoveTIC}}
\fancyfoot[RO, LE]{\footnotesize \thepage}
\renewcommand{\headrulewidth}{0.5pt}
\renewcommand{\footrulewidth}{0.5pt}

\begin{document}
\maketitle

In time-series analysis, the well-known dynamic regression models allow formally modelling the dependence between a set of covariates and a dependent variable considering the intrinsic temporal component of all participant variables. Based on a previous study of \cite{cryer2008time} and \cite{hyndman2018forecasting}, a forward-selection method is proposed for adding new significant covariates from a given set to a regression model with their respective optimal lags.

Formally, dynamic linear regression models define the lineal dependence between a stochastic proccess $Y_t$ and a set of proccesses  $\mathcal{X}=\{ X_t^{(1)}, X_t^{(2)}, ..., X_t^{(m)}\}$ in times non-greater than $t$:

\[ Y_t = \beta_0 + \beta_1 X^{(1)}_{t-r_1} + \beta_2 X^{(2)}_{t-r_2} + \cdots + \beta_m  X^{(m)}_{t-r_m} + \eta_t \]

where $r_i \geq 0$, for $i=1,...,m$, and $\eta_t \sim$ ARMA(p,q).


\cite{cryer2008time} proposed a method named \textit{prewhitening} for removing spurious correlation between two processes  $X_t$ and $Y_t$ and, thus, cleanly detecting the existence of lineal dependency between them, as well as the optimal lag $r$ dependency occurs in. Following this methodology, our approach adds iteratively dependent processes to a model by checking if a significant correlation (following \cite{cryer2008time} methodology) exists between a new proccess and the residuals $\eta_t$ of the model.

Given a stochastic proccess $Y_t$ and a set of proccesses $\mathcal{X}=\{ X_t^{(1)}, X_t^{(2)}, ..., X_t^{(m)}\}$ which act as regressor variables in the model, and an information criterion for model evaluation, the method proceeds as follows:

\begin{enumerate}
    \item Search in $\mathcal{X}$ the proccess with its optimal lag that best simple regression model produces, based on the selected information criterion.
    \[ X^{\text{best}}_t = \underset{X\in\mathcal{X}}{\arg\min} \Big\{ \text{criteria}\big( Y_t = \beta_0 + \beta_1 X_{t-r} + \eta_t\big)\Big\} \] 
    \item If $X^{\text{best}}_t$ exists, construct the model $\mathcal{M}: Y_t = \beta_0 + \beta_1 X^{\text{best}}_{t-r} + \eta_t$, remove  $X^{\text{best}}$ from $\mathcal{X}$ and proceed iteratively:
    \begin{enumerate}[label*=\arabic*.]
        \item Search in $\mathcal{X}$ the proccess  $X^{\text{best}}_t$ such that:
        \[ X^{\text{best}}_t = \underset{X\in\mathcal{X}}{\arg\min} \Big\{ \text{criteria}\big( \tilde{Y}_t = \beta_0 + \beta_1 X_{t-r} + \eta_t\big)\Big\} \] 
        restricted to $\text{criteria}\big(\tilde{Y}_t = \beta_0 + \beta_1 X_{t-r} + \eta_t\big) < \text{criteria}(\mathcal{M})$.
        \item In case of finding such model, consider a new $\mathcal{M}: \tilde{Y}_t = \beta_0 + \beta_1 X^{\text{best}}_t + \eta_t$ and $\tilde{Y}_t = \eta_t$ and return to (2.1). Otherwise, stop the iteration.
    \end{enumerate}
    \item The dynamic regression model is formed by the set of proccesses selected as $X^{\text{best}}_t$ in each iteration of the algorithm.

\end{enumerate}

This new proposal has been implemented and optimized in the statistical language \href{https://www.r-project.org/}{R} as a package, and it has been applied to multiple simulations to validate its performance. Finally, the obtained results from the IRAS database of Catalonia are presented to analyze the COVID-19 evolution.

\bibliography{bibliography}
\bibliographystyle{apalike}


\end{document}

