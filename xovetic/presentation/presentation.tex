\documentclass[10pt]{beamer}
\usetheme[secheader]{Boadilla}
\usepackage{amssymb,amsmath}
\usepackage[utf8]{inputenc}

\usepackage{xcolor}
\usepackage{hyperref}
\hypersetup{
    colorlinks = true,
    linkcolor = blue,
    urlcolor = blue
}

\usepackage[capitalise, noabbrev, nameinlink]{cleveref}
\usepackage{float}
\usepackage{booktabs}
\usepackage{makecell}

% -------- Portada ---------
\title{Automatic covariates selection in dynamic regression models with application to COVID-19 evolution}

\author{
    Ana Ezquerro \inst{1} 
    \and 
    Germán Aneiros \inst{2}
    \and 
    Manuel Oviedo de la Fuente \inst{3}
}
\institute{
   \inst{1} University of A Coruña, \texttt{ana.ezquerro@udc.es}
\and
    \inst{2} Grupo MODES, Departamento de Matemáticas, University of A Coruña, CITIC \\ \texttt{german.aneiros@udc.es}
\and
    \inst{3} Grupo MODES, Departamento de Matemáticas, University of A Coruña, CITIC \\ \texttt{manuel.oviedo@udc.es}
}

\date{Congreso XoveTIC 2022}

\begin{document}

\frame{\titlepage}

\begin{frame}
    \frametitle{Introduction}
        The \textbf{linear dynamic regression model} defines the linear dependence between a stochastic process $Y_t$ and a set of processes $\mathcal{X} =  \{ X_t^{(1)}, ..., X_t^{(m)} \}$:

        \[  Y_t = \beta_0 + \beta_1 X_{t-r_1}^{(1)} + \cdots \beta_m X_{t-r_m}^{(m)} + \eta_t
        \]

        constrained to $r_i\geq 0$ for $i=1,...,m$ and $\eta_t \sim \text{ARMA}(p,q)$\footnote{here we denote the well-known AutoRegressive Moving Average model as ARMA.}.



        \begin{itemize}
            \item \cite{cryer2008time} proposed the \textit{prewhitening} as a technique for removing spurious correlation between processes in order to detect linear correlation.
            \item We propose a forward-selection method that iteratively adds regressor variables (from a set of candidates) $Y_t$ is \textit{significantly} dependent with (assuming $Y_t$ is known and fixed). 
            \item Widespread application in financial, economic, political and biomedical fields.
            \item \textbf{Procedure}: analyze the mutual impact of several variables in order to define mathematical relationships and use them in forecasting.
        \end{itemize}

\end{frame}

\begin{frame}{Prewhitening}

    \cite{cryer2008time} states a linear correlation between two processes $X_t$ and $Y_t$ exists for some $k\leq 0$ if and only if 
    
    \[ \rho_k(\ddot{X}_t, \ddot{Y}_t) \qquad \text{is statistically significant,} \] 

    \begin{itemize}
        \item $\rho_k(X_t^{(1)}, X_t^{(2)})$ is the cross-correlation function between processes $X_t^{(1)}$ and $X_t^{(2)}$ lagged $k$ moments.
        \item $\ddot{X_t}$ and $\ddot{Y}_t$ are obtained via some linear filter application to $X_t$ and $Y_t$, respectively, ensuring one of them is white noise and the other is a stationary process (\textit{prehitening}).
    \end{itemize}

    \begin{block}{}
        Our proposal:
        \begin{itemize}
            \item Iteratively adds in the model significant covariates with their respective lags that optimize some information criterion (IC) of the resultant residuals.
            \item Uses the \textbf{residuals} of the last model created (by adding covariates) to check the existence of correlation between the dependent variable and a \textbf{new covariate candidate}.
        \end{itemize}
    \end{block}

\end{frame}

\begin{frame}{Example of spurious correlation and prewhitening}
    \begin{figure}
        \includegraphics[scale=0.35]{example prewhitening.pdf}
    \end{figure}
    
\end{frame}

\begin{frame}{Methodology}
    

    Let $Y_t$ be the dependent variable and $\mathcal{X}$ the set of covariates candidates. Thus, selection proceeds as follows:
    
    \begin{enumerate}
        \item Initialization. Consider $X_t^\text{best}$ as the variable (lagged $r$ moments) that minimizes de IC of the constructed model with $Y_t$:
        \[ X_t^\text{best} = \underset{X_t\in\mathcal{X}}{\arg\min} \Bigg\{ \text{IC}\Bigg( Y_t = \beta_0 + \beta_1 X_{t-r}^\text{best} + \eta_t \Bigg)\Bigg\} \] 
        \item Iteration. Use the regression errors ($\eta_t$) of the last model created to check if some correlation exists between the rest of the covariates not yet added to the model. Find again the ``best'' variable and add it to the model to obtain a new IC value. If this value improves the last one achieved, repeat this step. If it does not, stop the iteration.
        \item Finalization. The errors of the last fitted model must satisfy the stationary property. In other case, consider the regular differentiation of all data and start again the procedure.
    \end{enumerate}
\end{frame}

\begin{frame}{Example of the iterative selection step by step}


    \begin{table}
        \centering\small
        \setlength{\tabcolsep}{5pt}
        \caption{From \href{https://www.worldometers.info/coronavirus/}{worldmeters} source, model the evolution of \textit{exitus} cases in Spain due to the COVID-19 considering other country data} 
        \label{covid19}
    
        \begin{tabular}{|l|ccc|}
            \hline
            \textbf{Covariate}           & \textbf{Lag}  & \textbf{Coefficient est. (s.e)} & \textbf{AICc}          \\ 
            \hline 
            \texttt{confirmed\_spain}    & 0             & 0.0064 (0.0009)                 & 5596.641             \\ 
            \texttt{recovered\_portugal} & -1            & -0.0337 (0.0063)                & 5550.963             \\
            \texttt{recovered\_france}   & 0             & 0.0646 (0.0142)                 & 5540.655             \\
            \texttt{confirmed\_france}   & 0             & -0.0033 (0.0007)                & 5522.699             \\
            \texttt{confirmed\_portugal} & -13           & 0.0395 (0.0085)                 & 5504.169             \\
            \texttt{recovered\_spain}    & -7            & -0.0669 (0.0176)                & 5500.573             \\ 
            \hline 
        \end{tabular}
    \end{table}    
    \begin{itemize}
        \item \textbf{Note}: This model was fitted with differentiated data.
        \item \texttt{deaths\_england}, \texttt{deaths\_france}, \texttt{confirmed\_england}, \texttt{recovered\_england} and \texttt{deaths\_portugal} were not included in the model.
        \item The residuals of the model follow an ARMA$(1,1)\times (1,1)_7$ with parameters:
        \[
        \begin{array}{l}
            \phi_1=0.9724 (0.0131), \ \theta_1=-0.7508 (0.0370), \\
            \Phi_1=0.6958 (0.1892), \ \Theta_1=-0.5512 (0.2215)
        \end{array}
        \] 
    \end{itemize}
\end{frame}


\begin{frame}{Results of multiple simulations}

    
    
\end{frame}

% add figure about iteratively


\begin{frame}
    \bibliography{bibliography}
    \bibliographystyle{apalike}
\end{frame}

\end{document}  